\documentclass{article}

\pdfcompresslevel=0
\usepackage{listings}

\title{This PDF is an OCaml bytecode}
\date{}

\begin{document}

\begingroup\relax
\immediate\pdfobj stream
attr {/Type /EmbeddedFile} file {clean.byte}
\immediate\pdfobj{<<
  /Type /Filespec
  /F (thing.byte) /EF <</F \the\pdflastobj\space 0 R>>
  >>}
\pdfannot{
  /Subtype /FileAttachment /FS \the\pdflastobj\space 0 R
  /F 2 % Flag: Hidden
}
\endgroup

\maketitle

This PDF is an OCaml bytecode.
You can execute it:
\begin{lstlisting}
% ocamlrun thing.pdf
\end{lstlisting}

If you use an appropriately modern PDF viewer, this PDF also \emph{contains}
an OCaml bytecode \cite{pocgtfo4}.

\begin{thebibliography}{9}

\bibitem{pocgtfo4}
  {POC$\|$GTFO, Issue 4},
  27 June 2014.

\end{thebibliography}

\end{document}



%%% Local Variables:
%%% mode: latex
%%% TeX-master: t
%%% End:
